This thesis presents a measurement of integrated, single- and double-differential production cross-sections of the W boson decaying into a muon and a neutrino at the Large Hadron Collider. The analysis utilizes the full 2011 dataset collected by the ATLAS experiment, with an integrated luminosity of \lumitr.

A good understanding of the detector yielded a sub-percent level of precision on the fiducial integrated and single-differential cross-section as a function of $|\eta|$ (excluding the luminosity uncertainty). This represents an improvement by more than a factor of two with respect to the 2010 measurement and leaves experimental uncertainties comparable in size to those on the state-of-the-art theoretical predictions.

Comparison of the data with NNLO predictions reveals that existing families of parton distribution functions provide various levels of agreement with the data, with tensions arising in some regions of phase space for some of the PDF families. Initial studies to leverage these tensions to constrain the parton distribution functions with the 2011 $W$ and $Z/\gamma^{*}$ measurements suggest a substantial reduction in the down and strange quark PDF uncertainties.

The double-differential measurement in muon $|\eta|$ and $p_T$ is the first of its kind and features a 2-4\% uncertainty in most $p_T$ slices. Disagreements with NNLO predictions are observed, particularly above the Jacobian peak, suggesting that the soft effects normally handled through parton showering are not well-described by fixed-order calculations.
