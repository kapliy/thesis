% acknowledgements

First and foremost, I would like to thank my adviser Mel Shochet. With Mel's guidance, support, and funding, I have worked on several exciting projects at Chicago: from the software simulations of the FastTracker trigger, to the hardware and firmware development of a high-speed transceiver circuit board, to a precision-electroweak analysis measuring the differential cross-sections of the \Wpm\ bosons. Mel combines several decades of experience in all facets of high energy physics research with an amazing intuition: whenever I approached him with a problem, I always received a well-thought advice that was right on the money.

In addition to Mel, I am grateful to David Biron, Yau Wah, and Liantao Wang, who agreed to serve on my Ph.D. committee.

I thank my colleagues at CERN and at various European universities, including Massimiliano Bellomo, Jan Kretzschmar, Adrian Lewis, and many others. This analysis was a collective effort by the entire W/Z-inclusive team.

The high-energy physics group at Chicago has been a valuable asset since my first day at the university. From private discussions, to seminars and conferences, I have learned a lot from the professors and other group members at Chicago and have benefited greatly from a friendly and nurturing environment.

Several post-docs shared their expertise and wisdom, including John Alison, Antonio Boveia, Erik Brubaker, Bjoern Penning, Lauren Tompkins, Joe Tuggle, Guido Volpi, and Kohei Yorita. Peter Onyisi provided so much input at all stages of this analysis that he qualified as the co-author.

A number of HEP students shared with me the joys and challenges of graduate-student life, including Yangyang Cheng, Tudor Costin, Jeff Dandoy, Eric Feng, Karol Krizka, Shawn Kwang, Ho Ling Li, Sam Meehan, Constantinos Melachrinos, Sasha Paramonov, Jian Tang, Jordan Webster, Daping Weng, and Zihao Zhang.

The computing resources available at Chicago allowed me to run the analysis through multiple iterations in a quick and efficient manner. My thanks go to Rob Gardner and his team: Lincoln Bryant, David Champion, Marco Mambelli, Aaron van Meerten, Charles Waldman, Ilija Vukotic, and Nathan Yehle.

Mary Heinz provided first-class IT support throughout the years. She often went above and beyond the call of duty - from quick responses to my weekend pleas to reboot a server that went down, to setting up obscure Linux device drivers.

For a couple of years, I worked in the electronics shop side-by-side with Mircea Bogdan, Jean-Francois Genat, Harold Sanders, and Mark Zaskowski. Fukun Tang and I had a lot of fun designing and building nearly three hundred data transmission cards and shared many conversations in the process.

In my first year at Chicago, I was a teaching assistant for the introductory physics classes and received much-appreciated guidance from Van Bistrow and Stuart Gazes. Autym Henderson, Nobuko McNeill, and Aspasia Sotir-Plutis helped me navigate through the bureaucratic rules at the university and ensured that I always got paid on time and received reimbursement for my travel expenses.

My parents Sergey and Liubov, my brother Denis, and my extended family in Russia and China have provided unyielding love and support throughout graduate school. My gratitude to them is boundless.

I met Meng Ning in a physics lab in KPTC during my first year of graduate school. Fast-forward to today, and we are about to celebrate our fifth wedding anniversary. Meng has always been there for me, and I am very excited as the two of us, along with our cats, are opening a new page in our lives in Boston.
