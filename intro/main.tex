% introduction, thesis outline

\section{Executive Summary}

The Large Hadron Collider (LHC) is the world's largest and highest-energy particle accelerator. Located on the border of Switzerland and France, it collides two energetic proton beams, creating sub-atomic particles that are hard to produce in other ways. Among these are the positive and negative \Wboson\ bosons, which are the mediators of the weak nuclear force. Although \Wboson\ bosons decay quickly before reaching the detector, their decay products can often be identified and measured experimentally. This thesis measures the propensity of \Wboson\ decay products to appear at different angles or energies, i.e. differential cross-sections, which provides a precise test of perturbative Quantum Chromodynamics (QCD) and a validation of theoretical predictions at next-to-next-to-leading order (NNLO). These measurements also provide valuable information about the structure of the proton, which is not calculable from the first principles.

The analysis uses the full 2011 dataset collected from the ATLAS experiment, one of two general-purpose detectors at the LHC. The ATLAS detector is designed to measure the energy and direction of the particles that emanate from the collision point. This thesis focuses on the \Wpmmn\ decay channel, where the muon is detected explicitly in the inner tracker and muon chambers, while the non-interacting neutrino is inferred through the presence of momentum imbalance in the detector. A complementary measurement in the \Wpmen\ channel was simultaneously performed by a different team and combined with the results obtained in this thesis.

The measurement is considered ``inclusive'' in the sense that it does not distinguish between final states with different numbers of jets - collimated sprays of hadrons that may be produced along with the \Wboson\ boson. These multi-jet final states are explored in a separate $W+jets$ analysis~\cite{Aad:2012en}.

\section{Thesis Outline}

During 2011, the LHC operated at the center-of-mass energy of $7\tev$, about half of its maximum design energy, but substantially higher than the $2\tev$ accessible in the recently completed Tevatron program at Fermilab. The ATLAS and CMS collaborations previously published analyses of the inclusive \Wboson\ production cross-sections at $7\tev$ using the 2010 dataset, with an integrated luminosity of \lumioldtr~\cite{Aad:2011dm,CMS:2011aa}. This thesis extends those measurements to the full 2011 sample with \lumitr, a more than 100-fold increase in integrated luminosity. The increased statistics along with improved understanding of the detector resulted in a significant decrease in experimental measurement uncertainties from 2\% to 1\% in the integrated and single-differential (angular variable) cross-sections. In addition, this thesis provides the first measurement of the double-differential cross-sections binned in both the angular variable and the transverse momentum of the muon.

Chapter~\ref{chap:method} explains the methodology for cross-section calculation and defines the phase space and binning used in each part of the measurement. Chapter~\ref{chap:det} introduces the LHC and the key components of the ATLAS detector. Chapter~\ref{chap:th} describes the theory behind \Wboson\ boson production at hadron colliders and outlines the process of obtaining theoretical predictions. Chapter~\ref{chap:mc} describes the data and Monte-Carlo samples used in the analysis. Chapter~\ref{chap:perf} focuses on the reconstruction performance and data-driven corrections for detector defects.

Chapter~\ref{chap:evt} describes experimental selection cuts and shows various kinematic distributions of the selected events. Chapter~\ref{chap:bg} reviews the backgrounds to the measurement and estimates their composition and uncertainty. Chapter~\ref{chap:unc} summarizes all experimental uncertainties for the integrated, single, and double-differential measurements. Chapter~\ref{chap:res} contains the numerical results of the analysis, comparisons with theory predictions, and newly-derived constraints on the parton distribution functions.

Chapter~\ref{chap:conc} summarizes the measurement and draws conclusions.
